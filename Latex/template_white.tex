
% TASHUB LATEX TEMPLATE
% 
\documentclass[LaTeX2e,10pt,aspectratio=169]{beamer}
\usefonttheme[onlymath]{serif}

\usepackage[numbers]{natbib}
\usepackage{multimedia}
\usepackage{xcolor}
\usepackage{graphicx}


\definecolor{sotonMB}{rgb}{0 0.3594 0.5156} % soton marine blue (7469)
\definecolor{sotonCB}{rgb}{0.1 0.6844 0.7703} % soton cyan blue (3145)
\definecolor{sotonLB}{rgb}{0 0.5938 0.7617} % soton light blue (313)
\definecolor{sotonDG}{rgb}{0.3164 0.3828 0.4336} % soton dark grey (7545)
\definecolor{sotonGR}{rgb}{0.6367 0.5653 0.4180} % soton green (5777)
\definecolor{sotonME}{rgb}{0.7305 0.7305 0.7305} % soton metal (877)
\definecolor{sotonLG}{rgb}{0.6406 0.6797 0.7070} % soton light grey (7543)

\definecolor{sotonyel}{rgb}{.99999 .70196 .00000} % soton yellow
\definecolor{sotonora}{rgb}{.99608 .24314 .07843} % soton orange
\definecolor{sotonred}{rgb}{.94118 .05882 .17255} % soton red
\definecolor{sotonrus}{rgb}{.67059 .07059 .06275} % soton russet
\definecolor{sotonbrn}{rgb}{.54118 .25490 .16863} % soton brown
\definecolor{sotonpnk}{rgb}{.88627 .41176 .62353} % soton pink
\definecolor{sotonppl}{rgb}{.90 .90 .99} % soton purple

\setbeamertemplate{background canvas}[vertical shading][top=white,bottom=sotonppl]
\setbeamersize{text margin left=0mm,text margin right=1mm} 
\setbeamercolor{background canvas}{bg=}
\setbeamercolor{button border}{bg=sotonMB, fg=sotonCB}
\setbeamercolor{button}{bg=sotonMB, fg=DarkRed}



% COLORS AND BULLET POINT SETTINGS
\setbeamercolor{title page}{fg=purple}
\setbeamercolor{block title}{fg=sotonMB}
\setbeamercolor{frametitle}{fg=purple}
\setbeamercolor{framesubtitle}{fg=sotonMB}
\setbeamercolor{alerted text}{fg=sotonMB}
\setbeamercolor{normal text}{fg=white}
\setbeamercolor{titlelike}{fg=sotonMB}
\setbeamercolor{author}{fg=darkgray}
\setbeamercolor{date}{fg=darkgray}
\setbeamercolor{item}{fg=purple}
\setbeamercolor{caption name}{fg=darkgray}
\setbeamercolor{footline text}{fg=darkgray}
\setbeamertemplate{itemize item}[triangle]
\setbeamertemplate{itemize subitem}[triangle]
\setbeamertemplate{itemize subsubitem}[square]

% make title appear a bit lower
\addtobeamertemplate{frametitle}{\vskip10pt}{} 

% LOGOS
\graphicspath{{./Logos/}}
\newcommand{\putlogoTAS}{\includegraphics[width=3.5cm]{TAS-logo-CMYK-1_transparent.png}} % for TAS
\newcommand{\putlogoUni}{\includegraphics[width=3.5cm]{UOS_CMYK.pdf}} % for UoS

% TITLE PAGE
\setbeamertemplate{title page}{
	\vskip30pt
	\begin{centering}
		\vspace{0.5cm}
		\huge \inserttitle \\ \insertsubtitle 
		\small{
			\vskip20pt
			\insertauthor \\
			\vskip20pt
			\centering
			\insertdate \\~\\~\\
			}
	\end{centering}
}


\setlength{\unitlength}{1cm}




\author{Jane Doe}  % YOUR NAME HERE
\date{\today}      % DATE
\title{My Amazing Presentation} % TITLE

\begin{document}


% HEADLINES, FOOTLINES, AND BACKGROUND

% headline for the title page
\setbeamertemplate{headline}{
	\hspace{50pt}\putlogoTAS\hspace{50pt} % logo on the right
}

\frame{\titlepage}
\addtocounter{framenumber}{-1} % Do not count the title frame!

\setbeamertemplate{headline}{
	\vskip25pt % prevents logo added on normal pages
}

\setbeamertemplate{background} % ! SETS THE TASHUB LOGOS ON THE SLIDES
{\includegraphics[width=\paperwidth,height=\paperheight]{blankslide.png}}


\begin{frame}{Introductory slide}
% make as many nested bullet points as you want
\begin{minipage}{0.50\textwidth}  % use a minipage if you like two different columns, e.g. bullet points left and pictures right as here
\begin{itemize}			% bullet points left
\item Top-level bullet 1
\begin{itemize}
\item Bottom-level bullet point 1
\item Bottom-level bullet point 2
\end{itemize}
\item Top-level bullet 1
\begin{itemize}
\item Bottom-level bullet point 1
\item Bottom-level bullet point 2
\end{itemize} 
\end{itemize}
\end{minipage}
\begin{minipage}{0.26\textwidth} % pictures right
\centering
\vspace{0.15cm}
\fbox{
\parbox[c]{0.6\textwidth}{  % little frame around the figure
\includegraphics[width=0.6\textwidth]{example-image-a}  % figure
}
}\\
 some caption  \\
\vspace{0.15cm}
\fbox{
\parbox[c]{0.6\textwidth}{   % little frame around the figure
\includegraphics[width=0.6\textwidth]{example-image-b} % figure
}
}\\
 some other caption  \\
\end{minipage}
\end{frame}


\end{document}

